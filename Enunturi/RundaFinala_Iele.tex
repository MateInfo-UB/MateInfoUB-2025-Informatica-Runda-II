\documentclass[12pt,a4paper]{article}
\usepackage{task}

\newcommand{\contestName}{MateInfoUB}
\newcommand{\contestDates}{18 Mai 2025}
\newcommand{\contestPlace}{Facultatea de Matematic\v a-Informatic\v a}
\newcommand{\contestRound}{Runda Final\v a}
%\newcommand{\contestRound}{Rund\v a de Test}
\newcommand{\statementLanguage}{Rom\^ an\v a (Oficial)}
\newcommand{\taskShortName}{Iele}


\definecolor{Brown}{cmyk}{0,0.81,1,0.60}
\definecolor{OliveGreen}{cmyk}{0.64,0,0.95,0.40}
\definecolor{CadetBlue}{cmyk}{0.62,0.57,0.23,0}
\definecolor{lightlightgray}{gray}{0.9}

\lstset{
% language=C++,                             % Code langugage
basicstyle=\ttfamily,                   % Code font, Examples: \footnotesize, \ttfamily
keywordstyle=\bfseries,        % Keywords font ('*' = uppercase)
commentstyle=\color{gray},              % Comments font
columns=flexible,
% numbers=left,                           % Line nums position
% numberstyle=\tiny,                      % Line-numbers fonts
% stepnumber=1,                           % Step between two line-numbers
% numbersep=5pt,                          % How far are line-numbers from code
% backgroundcolor=\color{lightlightgray}, % Choose background color
% frame=none,                             % A frame around the code
% tabsize=2,                              % Default tab size
% captionpos=b,                           % Caption-position = bottom
% breaklines=true,                        % Automatic line breaking?
% breakatwhitespace=false,                % Automatic breaks only at whitespace?
% showspaces=false,                       % Dont make spaces visible
% showtabs=true,                         % Dont make tabls visible
}


\setlength{\parskip}{0pt}

\begin{document}
\input{header.tex}

\section*{Creaturi 3: Ielele}


\begin{center}
\includegraphics[scale=0.15]{iele.jpg}
\end{center}

Ielele s-au întâlnit în clar de lună pentru horă. Dar vai, pădurea lor fermecată a fost tăiată de săteni, pentru lemne de foc. Furioase, ielele vor să se răzbune. 

\vspace{1em}

În poiana în care ielele trăiesc există $N$ sate, conectate prin $N - 1$ drumuri, în așa fel încât din oricare sat se poate ajunge în oricare alt sat. Fiecare drum are o lungime, exprimată ca un număr natural între $1$ și $M$.

\vspace{1em}

Ielele se întreabă: \textit{Câte parcursuri există între sate, pentru care lungimile drumurilor formează o permutare a numerelor de la $1$ la $M$?}

\vspace{1em}

Un \textit{parcurs} este o cale între două sate care urmează drumurile fără a trece de două ori prin același drum. Cu alte cuvinte, este o succesiune de drumuri care leagă două sate fără a face bucle sau a se întoarce. Un \textit{parcurs} este considerat egal cu parcursul invers.

\vspace{1em}

Ajutați-le să calculeze numărul de parcursuri posibile.

\subsection*{Date de intrare}

Pe prima linie se găsesc numerele $N$ și $M$, cu semnificația din enunț.

Pe următoarele $N - 1$ linii se găsesc câte 3 numere $A$, $B$ ($1 \leq A, B \leq N$) și $L$ ($1 \leq L \leq M$), semnificând că există un drum de lungime $L$ între satul $A$ și satul $B$.

\subsection*{Date de ieșire}

Pe unica linie afișati numărul cerut. 

\subsection*{Constrângeri}

\begin{itemize}
    \item $1 \leq N \leq 10^5$.
    \item $1 \leq M < N$.
    \item Se garantează că datele din input sunt corecte.
\end{itemize}


\subsection*{Subtask-uri}

\begin{enumerate}
    \item ($20$ de puncte) $1 \leq N \leq 1000$.
    \item ($20$ de puncte) $M = 2$.
    \item ($20$ de puncte) $M = 3$.
    \item ($20$ de puncte) Există un parcurs care trece prin toate satele.
    \item ($20$ de puncte) Nicio constrângere suplimentară.
\end{enumerate}

\subsection*{Exemplu}

\begin{tabular}{|@{}p{0.5\textwidth}@{}|@{}p{0.5\textwidth}@{}|}
\hline
\multicolumn{1}{|c|}{\bfseries Input Standard (\textit{cin})} &
\multicolumn{1}{c|}{\bfseries Output Standard (\textit{cout})} \\
\hline
\begin{textQuoteCell}
6 3
1 2 1
2 3 2
3 4 3
3 5 1
2 6 3
\end{textQuoteCell} &
\begin{textQuoteCell}
2
\end{textQuoteCell} \\    
\hline
\end{tabular}
\vspace{1em}

\subsection*{Explicație}

Există două parcursuri valide:
\begin{itemize}
    \item Parcursul \code{6 - 2 - 3 - 5}, și
    \item Parcursul \code{4 - 3 - 2 - 1}.
\end{itemize}

Nu există niciun alt parcus ale cărui lungimi de drumuri să formeze o permutare a numerelor de la $1$ la $3$.

\end{document}