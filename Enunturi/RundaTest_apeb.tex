\documentclass[12pt,a4paper]{article}
\usepackage{task}

\newcommand{\contestName}{MateInfoUB}
\newcommand{\contestDates}{17 Mai 2025}
\newcommand{\contestPlace}{Facultatea de Matematic\v a-Informatic\v a}
\newcommand{\contestRound}{Rund\v a de Test}
%\newcommand{\contestRound}{Rund\v a de Test}
\newcommand{\statementLanguage}{Rom\^ an\v a (Oficial)}
\newcommand{\taskShortName}{A/B}

\setlength{\parskip}{0pt}

\begin{document}
\input{header.tex}

\section*{A pe B}


Se dau două numere $A$ și $B$.

Trebuie să calculați raportul lor, rotunjit în jos. Altfel spus, aveți de calculat $[\frac{A}{B}]$.

% \subsection*{Detalii de Implementare}

\subsection*{Date de Intrare}


Pe prima linie a datelor de intrare se află numerele $A$ și $B$.

\subsection*{Date de Ieșire}

Afișați o singură linie, care să conțină raportul numerelor $A$ și $B$, rotunjit în jos.

\subsection*{Constrângeri}

\begin{itemize}
    \item $1 \leq A \leq 10^{50}$.
    \item $1 \leq B \leq 10^{50}$.
\end{itemize}


\subsection*{Subtask-uri}

\begin{enumerate}
    \item ($50$ de puncte) $1 \leq A, B \leq 10^{18}$.
    \item ($50$ de puncte) Nicio constrângere suplimentară.
\end{enumerate}

\subsection*{Exemplu}

\begin{tabular}{|@{}p{0.5\textwidth}@{}|@{}p{0.5\textwidth}@{}|}
\hline
\multicolumn{1}{|c|}{\bfseries Input Standard (\textit{cin})} &
\multicolumn{1}{c|}{\bfseries Output Standard (\textit{cout})} \\
\hline
\begin{textQuoteCell}
18 5
\end{textQuoteCell} &
\begin{textQuoteCell}
3
\end{textQuoteCell} \\    
\hline
\end{tabular}
\vspace{1em}

% In the first example, $n=5$. The answer is $0$ because...

% In the second example, $n=5$. The answer is $0$ because...

\end{document}